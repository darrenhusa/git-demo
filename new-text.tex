This is a demonstration of the advantages of this approach. The tools shown/described offer inherent advantages/capabilities that are not present with a word processor like Microsoft Word.
* Since the document is now stored in plain text, one can with the help of git and github software, place the text/document under source or version control
* One team member can work on the document from one or more computers through using git pull and git push to transfer the document from a local repository to a remote repository. With a free Github account all repositories are public. The Bitbucket site which operates similar to GitHub does allow private repositories with their free accounts.
* One can view "diffs" which are the actual changes between various commits or versions of the document. A diff shows additions and deletions between two different versions.
* One can git tag a particular version of the repository as important i.e. could constitute an offcial release of the Faculty Handbook document.
* One can convert the Latex tex file to a PDF using freely available software.
